%!TEX root = submission.tex

Historical events are difficult to define; historians and political scientists read large quantities of text to construct an accurate picture of a single ``event.''  Events are interesting by definition: they are the hidden causes of anomalous observations.  But they are also inherently abstract---we can observe that changes occur, but we cannot directly observe whether or not an event occurs.

\PP event detection is a real-world problem (why interesting?)

\PP what are events?
- they occur rarely
- deviate from normal behavior
- usually affect many datapoints (not just one)


% \PP outliers vs events \cite{Neill:2009} (univariate -> multivariate)
% How is event detection different from:
% 1. SupervisedLearning:
% • Abnormal events are extremely rare, normal events are
% plentiful
% 2. Clustering:
% • Clustering = partitioning data into groups
% • Not the same as finding statistically anomalous groups
% 3. OutlierDetection:
% • Events of interest are usually not individual outliers
% • The event typically affects a subgroup of the data rather than a single data point


- how does capsule define events?

\PP contibutitions and capsule name

\PP outline of the paper



% % Events occur in many sources of data: historical events can be identified from diplomatic messages, scientific events from publications, and network events from communications between users (such as email).  
% % Detecting events automatically is a well-studied problem, but approa

% % We present a model that detects when events occur and characterize.

% % Characterizing an event, however, can prove challenging.  

% \PP why are events interesting?

% \PP what is an event?  (how do we characterize it)

% \PP how can this construction be used? (why do we use the name ``Capsule?'')

% \PP contributions list (vis, model, code for both, exploration on historical corpus and arxiv/enron)

% \PP outline of remainder of paper

% \parhead{Related work.} 

% Automatic event detection is a well-studied problem.  \cite{Neill:2009}

% \PP automatic event detection approaches % see http://www.cs.cmu.edu/~neill/papers/eventdetection.pdf

% \PP topic modeling + viz (incl dynamic topic models)

% \PP network and related work there (e.g. hanna's work); say this isnt explicitly about netorks, but the data has this structure dn the model can be extended to use these concepts (do small experiment where ``entites'' are defined as to/from pairs)








\parhead{Related work.}  We first review previous work on automatic event detection and other related concepts.  

While Capsule uses text documents and associated metadata as input, event detection is often performed with univariate input data.  In this context, bursts that deviate from typical behavior (e.g., noisy constant or a repeating pattern) can define an event \cite{kleinberg2003bursty,ihler2007learning}; Poisson Processes~\cite{Kingman:1993} are often used to model events under this definition.  Alternatively, events can be construed as ``change points'' to mark when typical observations shift semi-permanently from one value to another~\cite{guralnik1999event}.
In both univariate and multivariate settings, the goal is often the same: analysts want to predict whether or not a rare events will occur~\cite{weiss1998learning,das2008anomaly}.  Capsule, in contrast, is designed to help analysts explore and understand the original data: our goal is interpretability, not prediction.

Text is often used in event detection, as it is an abundant source of data.  
In some applications, documents themselves are considered to be observed events~\cite{mccallum1998comparison,peng2007event}, or events are predetermined and tracked through the documents~\cite{yang2000improving,VanDam:2012}.  We are interested in detecting \emph{unobserved} events which can be characterized by patterns in the data.

A common goal is to identify clusters of documents; these approaches are used on news articles~\cite{zhao2012novel,zhao2007temporal,zhang2002novelty,li2005probabilistic,wang2007mining,allan1998line} and social media posts~\cite{VanDam:2012,lau2012line,jackoway2011identification,sakaki2010earthquake,reuter2012event,becker2010learning,sayyadi2009event}.  
In the case of news articles, the task is to create new clusters as novel news stories appear---this does not help disentangle typical content from rare events of interest.
Social media approaches identify rare events, but the methods are designed for short, noisy documents; they are not appropriate for larger documents that contain information about a variety of subjects.

Many existing methods use document terms as features, frequently weighted by tf-idf value~\cite{fung2005parameter,kumaran2004text,brants2003system,das2011dynamic,zhao2007temporal,zhao2012novel}; here, events are bursts in groups of terms.  Because language is high dimensional, using terms as features limits scalability.

Topic models~\cite{Blei:2012} reduce the dimensionality of text data; they have been used to help detect events mentioned in social media posts~\cite{lau2012line,dou2012leadline} and posts relevant to monitored events~\cite{VanDam:2012}.
We rely on topic models to characterize both typical content and events, but grouped observations can also be summarized directly~\cite{peng2007event,chakrabarti2011event,gao2012joint}.

In addition to text data over time, author~\cite{zhao2007temporal}, news outlet~\cite{wang2007mining}, and spatial information~\cite{Neill:2005,mathioudakis2010identifying,liu2011using} can be used to augment event detection.  Capsule uses author information in order to characterize typical concerns of authors.

Detecting and characterizing relationships~\cite{schein2015bayesian,linderman2014discovering,das2011dynamic} is related to event detection.  When a message recipient is known, Capsule's author input can be replaced with a sender-receiver pair, but the model could be further tailored for interactions within networks.

Once events have been identified and characterized, visualization translates a model's output into sometime intepretable for non experts.  Leadline~\cite{dou2012leadline} is an excellent example of a visualization of event detection.  We build on topic model visualization concepts~\cite{chaney2012visualizing} to provide tailored visualization code for Capsule.