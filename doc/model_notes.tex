% \RequirePackage[l2tabu, orthodox]{nag}
% \documentclass{article}

% % FONTS
\usepackage[T1]{fontenc}

% Replace default Latin Modern typewriter with its proportional counterpart
% http://www.tug.dk/FontCatalogue/lmoderntypewriterprop/
\renewcommand*\ttdefault{lmvtt}

%\usepackage[bitstream-charter]{mathdesign} % not needed for mtpro
\usepackage{amsmath}
\usepackage[subscriptcorrection,
            amssymbols,
            mtpbb,
            mtpcal,
            nofontinfo  % suppresses all warnings
           ]{mtpro2} %TODO: put htis back in and take out mathdesign
\usepackage{scalefnt,letltxmacro}
\LetLtxMacro{\oldtextsc}{\textsc}
\renewcommand{\textsc}[1]{\oldtextsc{\scalefont{1.10}#1}}
\usepackage[scaled=0.92]{PTSans}


% % COLOR
\usepackage[usenames,dvipsnames]{xcolor}
\definecolor{Green}{HTML}{156946}

% % SPACING and TEXT
\usepackage[final,expansion=alltext]{microtype}
\usepackage[english]{babel}
\usepackage{afterpage}
\usepackage{framed}


% define a paragraph header function
\DeclareRobustCommand{\parhead}[1]{\textbf{#1}~}

% % EDITING
% paragraph helper
\DeclareRobustCommand{\PP}{\textcolor{Plum}{\P} }

% % COUNTERS
\usepackage{enumitem}
\renewcommand{\labelenumi}{\color{black!67}{\arabic{enumi}.}}
\renewcommand{\labelenumii}{{\color{black!67}(\alph{enumii})}}
\renewcommand{\labelitemi}{{\color{black!67}\tiny$\blacksquare$}}
\renewcommand{\labelitemii}{{\color{black!67}\textbullet}}
\renewcommand{\labelitemiii}{{\color{black!67}\scriptsize$\blacktriangleright$}}

% FIGURES
\usepackage{graphicx}

% TABLES
\usepackage{booktabs}

% ALGORITHMS
\usepackage[algoruled]{algorithm2e}
\usepackage{listings}
\usepackage{fancyvrb}
\fvset{fontsize=\normalsize}

% % HYPERREF
\usepackage[colorlinks,linktoc=all]{hyperref} %TODO: rm draft
\usepackage[all]{hypcap}
\hypersetup{citecolor=Green}
\hypersetup{linkcolor=Green}
\hypersetup{urlcolor=Green}

% % CLEVEREF must come after HYPERREF
\usepackage[nameinlink]{cleveref}

% BIBLIOGRPHY: get rid of reference labels
\makeatletter
	\renewcommand\@biblabel[1]{}
\makeatother

% ACRONYMS
\usepackage[acronym,smallcaps,nowarn]{glossaries}
% \makeglossaries

% % COLOR DEFINITIONS
\newcommand{\green}[1]{\textcolor{Green}{#1}}


% LISTINGS DEFINTIONS
\lstdefinestyle{mystyle}{
    commentstyle=\color{Green},
    keywordstyle=\color{Green},
    numberstyle=\tiny\color{black!60},
    stringstyle=\color{Green},
    basicstyle=\ttfamily,
    breakatwhitespace=false,
    breaklines=true,
    captionpos=b,
    keepspaces=true,
    numbers=left,
    numbersep=5pt,
    showspaces=false,
    showstringspaces=false,
    showtabs=false,
    tabsize=2
}
\lstset{style=mystyle}

% ALGORITHMS
\usepackage[algoruled]{algorithm2e}

% FRAMES
\usepackage[linewidth=1pt]{mdframed}
\DeclareRobustCommand{\mb}[1]{\ensuremath{\boldsymbol{\mathbf{#1}}}}

\DeclareMathOperator*{\argmax}{arg\,max}
\DeclareMathOperator*{\argmin}{arg\,min}

\DeclareRobustCommand{\KL}[2]{\ensuremath{\textrm{KL}\left(#1\;\|\;#2\right)}}

\newcommand{\mbx}{\mathbold{x}}
\newcommand{\mbX}{\mbf{X}}

\newcommand{\mbz}{\mathbold{z}}

\newcommand{\mbI}{\mbf{I}}

\newcommand{\mbZ}{\mbf{Z}}
\newcommand{\mbL}{\mbf{L}}

\newcommand{\mbtheta}{\mathbold{\theta}}
\newcommand{\mbTheta}{\mathbold{\Theta}}
\newcommand{\mbomega}{\mathbold{\omega}}
\newcommand{\mbOmega}{\mathbold{\Omega}}
\newcommand{\mbsigma}{\mathbold{\sigma}}
\newcommand{\mbSigma}{\mathbold{\Sigma}}

\newcommand{\mblambda}{\mathbold{\lambda}}
\newcommand{\mbgamma}{\mathbold{\gamma}}
\newcommand{\mbzeta}{\mathbold{\zeta}}
\newcommand{\mbeta}{\mathbold{\eta}}
\newcommand{\mbbeta}{\mathbold{\beta}}
\newcommand{\mbphi}{\mathbold{\phi}}
\newcommand{\mbmu}{\mathbold{\mu}}
\newcommand{\mbrho}{\mathbold{\rho}}

\newcommand\dif{\mathop{}\!\mathrm{d}}
\newcommand{\diag}{\textrm{diag}}
\newcommand{\supp}{\textrm{supp}}

\newcommand{\E}{\mathbb{E}}
\newcommand{\Var}{\mathbb{V}\textrm{ar}}

\newcommand{\bbN}{\mathbb{N}}
\newcommand{\bbZ}{\mathbb{Z}}
\newcommand{\bbR}{\mathbb{R}}
\newcommand{\bbS}{\mathbb{S}}

\newcommand{\cL}{\mathcal{L}}

\newcommand{\cN}{\mathcal{N}}
\newcommand{\Gam}{\textrm{Gam}}
\newcommand{\InvGam}{\textrm{InvGam}}

\newcommand{\g}{~\vert~}
\newacronym{KL}{kl}{Kullback-Leibler}
\newacronym{ELBO}{elbo}{\emph{evidence lower bound}}
\newacronym{POPELBO}{pop-elbo}{\emph{population evidence lower bound}}

\newacronym{SVI}{svi}{stochastic variational inference}
\newacronym{BUMPVI}{bump-vi}{bumping variational inference}

\newacronym{GMM}{gmm}{Gaussian mixture model}
\newacronym{LDA}{lda}{latent Dirichlet allocation}

\newacronym{SUTVA}{sutva}{stable unit treatment value assumption}

\begin{document}

\title{Modeling Cables Events}

\author{Allison J.B. Chaney}

\maketitle

\section{Introduction}

Our objective is to detect historical events.

\section{Related Work}

- BBVI, DEFs; this deals with finding updates for non-conjuate models

- non uniform sampling (MMSB in PNAS); we want events to be represented
appropriately.  Also: how to update global stuff form non-uniform document
sampling--it's a bit of challenge)

- spike and slab (e.g., Emperical study of SVI for Beta Bernoulli Process; also
IBP papers) because we have variables that fit this pattern

- SSVI; this helps us with the event variables--they are bound together (is
this the real purpose here?)


\section{Generative model} We start with a fitted LDA model where documents
are represented in terms of topics ($\theta$, a $D \times K$ matrix), and
topics are represented as a distribution over words ($\beta$, a $K \times V$
matrix).  This model fit, along with document metadata, are our observations.
Alternatively, we can tack on the LDA generative process to the model below.

\begin{itemize}
\item for each day $i$ with date $a_i$:
\begin{itemize}
\item generate event occurrence/strength $\epsilon \sim \mbox{Poisson}
(\eta_\epsilon)$, where $\eta_\epsilon$ is a fixed, non-negative
hyperparameter for the mean event strength
\item generate the day/event's description in terms of each topic $k$: $\pi_{ik} \sim \mbox{Gamma}(\alpha_0, \beta_0)$, where $\alpha_0$ and $\beta_0$ are fixed hyperparameters.
\end{itemize}
\item draw the entity's base topics: $\phi_{0k} \sim \mbox{Gamma}(\alpha, \beta)$ (eventually for each entity, but for now, just limit data to only one entity)
\item For each cable $j$ on date $c_j$:
\begin{itemize}
	\item set cable topic parameter: $\phi_{jk} = \phi_{0k} + \sum_i f(a_i, c_j) \pi_{ik} \epsilon_i$, where $f$ is defined below.
	\item draw cable topic: $\theta_{jk} \sim \mbox{Gamma}(\beta_c \phi_{jk}, \beta_c)$
\end{itemize}
\end{itemize}

\noindent Note that\[f(a, c) =
\begin{cases}
	1 - \frac{c-a}{d}, & \mbox{if } a \le c < a+d \\
	0, & \mbox{otherwise,}
\end{cases} \]
where $d$ is the time distance (in days) after event $a$ at which point the event is no longer relevant.


\section{Inference}
For now, we assume that we know the LDA topics $\beta$ and only observe the documents in terms of their topics $\theta$; breaking this assumption makes inference a little more complicated as the updates for $\theta$ would have new dependencies.  Pending the results, we should explore that vein. 

Here, we follow the structure of the DEF paper to explain inference for this model.

\begin{itemize}
\item As usual, inference is the central computational problem.
\item Variational inference minimizes the KL divergence from an approximating distribution $q$ to the true posterior $p$.
\item This is equivalent to maximizing the ELBO: $\mathcal{L}(q)  = \E_{q(\epsilon, \pi, \phi)}[\log p(\theta,\epsilon,\pi,\phi) - \log q(\epsilon, \pi, \phi)]$
\item we define the approximating distribution $q$ using the mean field assumption: $q(\epsilon, \pi, \phi) = \prod_i q(\epsilon_{i})\prod_k\left[q(\phi_{0k})\prod_i q(\pi_{ik})\right]$ 
\item $q(\pi)$ and $q(\phi)$ are both gamma-distributed with variational parameters $\lambda^\pi$ and $\lambda^\phi$, respectively, where we use the softmax function $\mathcal{M}(x) = \log(1+\exp(x))$ to constrain the free variational parameters.
$q(\epsilon)$ is Poisson-distributed with variational parameter $\lambda^\epsilon$; the free parameter is also constrained by the softmax function.

\item the expectations under $q$ (needed to maximize the ELBO) do not have a simple analytic form, so we use ``black box'' VI techniques
\item for each variable, we can write the log probability of all terms containing that variable, giving us 
\[\log p^\epsilon_{i}(\theta, \epsilon, \pi, \phi) \triangleq \log p(\epsilon_{i} \g \eta_\epsilon) + \sum_{j : f(a_i, c_j) \neq 0}\sum_k\log p(\theta_{jk} \g \phi_{0k}, c_j, a_i, d, \beta_c, \pi_{ik}, \epsilon_i),\]
\[\log p^\pi_{ik}(\theta, \epsilon, \pi, \phi) \triangleq \log p(\pi_{ik} \g \alpha_0, \beta_0) + \mathbf{1}[\epsilon_i \neq 0] \sum_{j : f(a_i, c_j) \neq 0}\log p(\theta_{jk} \g \phi_{0k}, c_j, a_i, d, \beta_c, \pi_{ik}, \epsilon_i),\]
and
\[p^\phi_{k}(\theta, \epsilon, \pi, \phi) \triangleq \log p(\phi_{0k} \g \alpha, \beta) + \sum_{j} \log p(\theta_{jk} \g \phi_{0k}, c_j, a_i d, \beta_c, \pi_{ik}, \epsilon_i).\]
\item Then we can write the gradients with respect to the variational parameters as:
\[\nabla_{\lambda^\epsilon_{i}} \mathcal{L} = \E_q \left[ \nabla_{\lambda^\epsilon_{i}} \log q(\epsilon_{i} \g \lambda^\epsilon_{i}) \left( \log p^\epsilon_{i}(\theta, \epsilon, \pi, \phi) - \log q(\epsilon_{i} \g \lambda^\epsilon_{i}) \right)\right],\]
\[\nabla_{\lambda^\pi_{ik}} \mathcal{L} = \E_q \left[ \nabla_{\lambda^\pi_{ik}} \log q(\pi_{ik} \g \lambda^\pi_{ik}) \left( \log p^\pi_{ik}(\theta, \epsilon, \pi, \phi) - \log q(\pi_{ik} \g \lambda^\pi_{ik}) \right)\right],\]
and
\[\nabla_{\lambda^\phi_{k}} \mathcal{L} = \E_q \left[ \nabla_{\lambda^\phi_{k}} \log q(\phi_{0k} \g \lambda^\phi_{k}) \left( \log p^\phi_{k}(\theta, \epsilon, \pi, \phi) - \log q(\phi_{0k} \g \lambda^\phi_{k}) \right)\right].\]
\end{itemize}

Using this framework, we construct our black box algorithm below. 

\paragraph{For Reference}
The gamma distribution and derivatives:
\begin{align}
\log\mbox{Gamma}(x \g a,b) & = ab\log b - \log\Gamma \left(ab\right) +(ab-1)\log x - bx \label{eq:gamma}\\
%\log\mbox{Gamma}(x \g \mathcal{M}(a), \mathcal{M}(b)) & = 
%\mathcal{M}(a)\mathcal{M}(b)\log \mathcal{M}(b) - \log\Gamma \left(\mathcal{M}(a)\mathcal{M}(b)\right) +(\mathcal{M}(a)\mathcal{M}(b)-1)\log x - \mathcal{M}(b)x \\
\nabla_a \log\mbox{Gamma}(x \g \mathcal{M}(a), \mathcal{M}(b)) &= 
\mathcal{M}'(a) \mathcal{M}(b) \left[ \log \mathcal{M}(b) - \Psi \left(\mathcal{M}(a)\mathcal{M}(b)\right) + \log x\right] \label{eq:dgammaA}
\\
\nabla_b \log\mbox{Gamma}(x \g \mathcal{M}(a), \mathcal{M}(b)) &= 
\mathcal{M}'(b) \left[ \mathcal{M}(a) \left((\log \mathcal{M}(b) + 1) - \Psi \left(\mathcal{M}(a)\mathcal{M}(b)\right) + \log x\right) - x\right]\label{eq:dgammaB}
%\frac{d}{d b}\log\mbox{Gamma}(x \g \mathcal{M}(a), \mathcal{M}(b)) &= 
%\mathcal{M}'(b) \left[ \mathcal{M}(a) (\log \mathcal{M}(b) + 1) - \mathcal{M}(a)\Psi \left(\mathcal{M}(a)\mathcal{M}(b)\right) + \mathcal{M}(a)\log x - x\right]
\end{align}

The Poisson distribution and derivative:
\begin{align}
\log\mbox{Poisson}(x~\vert~\lambda) &= x\log\lambda - \log(x!) - \lambda \label{eq:poisson} \\
%\log\mbox{Poisson}(x~\vert~\mathcal{M}(\lambda)) &= x\log\mathcal{M}(\lambda) - \log(x!) -\mathcal{M}(\lambda) \\
\nabla_\lambda \log\mbox{Poisson}(x~\vert~\mathcal{M}(\lambda)) &= \mathcal{M}'(\lambda)\left[\frac{x}{\mathcal{M}(\lambda)} -1\right].\label{eq:dpoisson}
\end{align}

The softmax function and derivative:
\[\mathcal{M}(x) = \log(1+e^x)\]
\[\mathcal{M}'(x) = \frac{e^x}{1+e^x}\]
%\SetCommentSty{mycommfont}

\begin{algorithm}
\small
\DontPrintSemicolon
\KwIn{document topics $\theta$}
\KwOut{estimates of latent parameters entity topics $\phi$, event topics $\pi$,
and event occurances $\epsilon$}
\textbf{Initialize} $\lambda^\pi$, $\lambda^\phi$, and $\lambda^\epsilon$ to
respective priors \;
\textbf{Initialize} iteration count $t = 0$ \;
\While {change in validation likelihood $< \delta$}{
	initialize $\sigma^\pi = 0$ \;
	\For {each sample $s = 1, \dots, S$}{
		\For {each component $k$} {
			draw sample entity topics $\phi_{0k}[s] \sim \mbox{Gamma}(\mathcal{M}(\lambda^\phi_{k}))$ \;
			\BlankLine
			set $p^\phi_k[s] = \log p(\phi_{0k}[s] \g \alpha, \beta)$ \tcp*{see Eqn.~\ref{eq:gamma}}
			set $q^\phi_k[s] = \log q(\phi_{0k}[s] \g \lambda^\phi_k)$ \tcp*{Eqn.~\ref{eq:gamma} with params $\mathcal{M}(\lambda^{\phi}_k)$}
			set $g^\phi_k[s] = \nabla_{\lambda^\phi_{k}} \log q(\phi_{0k}[s] \g \lambda^\phi_{k})$ \tcp*{see Eqns.~\ref{eq:dgammaA}, \ref{eq:dgammaB}}
		}
		\For {each timestep $i$} {
			draw sample event occurance $\epsilon_i[s] \sim \mbox{Poisson}(\mathcal{M}(\lambda^\epsilon_i))$ \;
			\BlankLine
			set $p^\epsilon_i[s] = \log p(\epsilon_{i}[s] \g \eta)$ \tcp*{see Eqn.~\ref{eq:poisson}}
			set $q^\epsilon_i[s] = \log q(\epsilon_i[s] \g \lambda^\epsilon_i)$ \tcp*{Eqn.~\ref{eq:poisson} with param $\mathcal{M}(\lambda^{\epsilon}_i)$}
			set $g^\pi_{i}[s] = \nabla_{\lambda^\epsilon_{i}} \log q(\epsilon_{i}[s] \g \lambda^\epsilon_{i})$ \tcp*{see Eqn.~\ref{eq:dpoisson}}
			\BlankLine
			\If {$\epsilon_i[s] \neq 0$} {
				\For {each component $k$} {
					draw sample event topics $\pi_{ik}[s] \sim \mbox{Gamma}(\mathcal{M}(\lambda^\pi_{ik}))$ \;
					\BlankLine
					set $p^\pi_{ik}[s] = \log p(\pi_{ik}[s] \g \alpha_0, \beta_0)$ \tcp*{see Eqn.~\ref{eq:gamma}}
					set $q^\pi_{ik}[s] = \log q(\pi_{ik}[s] \g \lambda^\pi_{ik})$ \tcp*{Eqn.~\ref{eq:gamma} with params $\mathcal{M}(\lambda^{\pi}_{ik})$}
					set $g^\pi_{ik}[s] = \nabla_{\lambda^\pi_{ik}} \log q(\pi_{ik}[s] \g \lambda^\pi_{ik})$ \tcp*{see Eqns.~\ref{eq:dgammaA}, \ref{eq:dgammaB}}
				}
			}
		}
	}
	
	\For {each document $j$ (sent on date $c_j$ and has topics $\theta_j$), sample $s$ and component $k$}{
		set $\phi_{jk}[s] = \phi_j[s] + \sum_i f(a_i, c_j) \epsilon_i[s] \pi_{ik}[s]$ \;
		set $p^\theta_{jk}[s] = \log p(\theta_{jk} \g \phi_{jk}[s], \beta_c)$ \tcp*{see Eqn.~\ref{eq:gamma}}
		\BlankLine
		$p^\phi_k[s] \pluseq p^\theta_{jk}[s]$ \;
		\For {each timestep $i$ where $a_i \le c_j < a_i + d$}{
			$p^\epsilon_{i}[s] \pluseq \sum_k p^\theta_{jk}[s]$ \;
			\If {$\epsilon_i[s] \neq 0$} {
				$p^\pi_{ik}[s] \pluseq p^\theta_{jk}[s]$ \;
				update $\sigma^\pi_i \pluseq 1$ \;
			}
		}
	}

	
	set $\hat\nabla_{\lambda^{\phi} }\mathcal{L} \triangleq \frac{1}{S} \sum_s g^\phi[s] ( p^\phi[s] -  q^\phi[s] )$ \;
	set $\hat\nabla_{\lambda^{\epsilon}} \mathcal{L} \triangleq \frac{1}{S} \sum_s g^\epsilon[s] ( p^\epsilon[s] -  q^\epsilon[s] )$ \;
	set $\hat\nabla_{\lambda^{\pi}} \mathcal{L} \triangleq \frac{1}{\sigma_\pi} \sum_s g^\pi[s] ( p^\pi[s] -  q^\pi[s] )$ \;

	\BlankLine
	set $\rho= (t +\tau)^\kappa$ \;
	% update event content for each event $i$ and topic $k$: 
	set $\lambda^{\pi} \pluseq \rho \hat\nabla_{\lambda^{\pi}} \mathcal{L}$ \;
	set $\lambda^{\epsilon} \pluseq \rho \hat\nabla_{\lambda^{\epsilon}} \mathcal{L}$ \;
	set $\lambda^{\phi} \pluseq \rho \hat\nabla_{\lambda^{\phi}} \mathcal{L}$ \;
}

set $\E[\pi] = \lambda^{\pi,a}$ \;
set $\E[\phi] = \lambda^{\phi,a}$ \;
set $\E[\epsilon] = \lambda^{\epsilon}$ \;
\BlankLine
\Return{$\E[\pi]$, $\E[\phi]$, $\E[\epsilon]$} \;
\caption{Inference for Cables Model}
\label{alg:cables}
\end{algorithm}



\end{document}
