\documentclass[12pt]{article}

\renewcommand{\baselinestretch}{1.5}

\usepackage{bookman}
\usepackage[bitstream-charter]{mathdesign}

\begin{document}

\begin{center}
  {\large \textbf{Who, What, When, Where, and Why?
  \\ 
  A Computational Approach to
Understanding Historical Events Using State Department Cables}}\\
  \bigskip
  {Allison J.B.~Chaney,
  Hanna Wallach,
  David M.~Blei
  }
\end{center}


We develop computational methods for analyzing historical documents to
identify events of potential historical significance. Significant
events are characterized by interactions between entities (e.g.,
countries, organizations, individuals) that deviate from typical
interaction patterns. When studying historical events, historians and
political scientists commonly read large quantities of text to
construct an accurate picture of who, what, when, and where---a
necessary precursor to answering the more nuanced question, ``Why?'' Our
methods help historians identify possible events from the texts of
historical communication. Specifically, we build on topic modeling to
distinguish between topics that describe ``business-as-usual'' and
topics that deviate from these patterns, where deviations are also
indicated by particular entities interacting during particular periods
of time. To demonstrate our methods, we analyze a corpus of 2 million
State Department cables from 1973 to 1977. For example, we show that
we are able to detect and characterize the Fall of Saigon.


\subsection*{Keywords}
topic modeling, characterizing events, computational history,
government transparency and secrecy, declassified documents

\end{document}






































